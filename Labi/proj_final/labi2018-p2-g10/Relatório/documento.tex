\documentclass{report}
\usepackage[T1]{fontenc} % Fontes T1
\usepackage[utf8]{inputenc} % Input UTF8
%\usepackage[backend=biber, style=ieee]{biblatex} % para usar bibliografia
\usepackage{csquotes}
\usepackage[portuguese]{babel} %Usar língua portuguesa
\usepackage{blindtext} % Gerar texto automaticamente
\usepackage[printonlyused]{acronym}
\usepackage{hyperref} % para autoref
\usepackage{graphicx}

\bibliography{bibliografia}


\begin{document}
%%
% Definições
%
\def\titulo{Projeto Final}
\def\data{14 de Junho de 2018}
\def\autores{António Domingues, Diogo Fernandes,Pedro Alves, João Mota}
\def\autorescontactos{(89007) antonioccdomingues@ua.pt, (89221) diogo.fernandes77@ua.pt, (88861) pedroalves99@ua.pt,(89160)joaormota@ua.pt}

\def\versao{}
\def\departamento{Departamento de Eletrónica, Telecomunicações e Informática}
\def\empresa{Universidade de Aveiro}
\def\logotipo{ua.pdf}
%
%%%%%% CAPA %%%%%%
%
\begin{titlepage}

\begin{center}
%
\vspace*{50mm}
%
{\Huge \titulo}\\ 
%
\vspace{10mm}
%
{\Large \empresa}\\
%
\vspace{10mm}
%
{\LARGE \autores}\\ 
%
\vspace{30mm}
%
\begin{figure}[h]
\center
\includegraphics{\logotipo}
\end{figure}
%
\vspace{30mm}
\end{center}
%
\begin{flushright}
\versao
\end{flushright}
\end{titlepage}

%%  Página de Título %%
\title{%
{\Huge\textbf{\titulo}}\\
{\Large \departamento\\ \empresa}
}
%
\author{%
    \autores \\
    \autorescontactos
}
%
\date{\data}
%
\maketitle

\pagenumbering{roman}

%%%%%% RESUMO %%%%%%
\renewcommand{\abstractname}{Resumo}
\begin{abstract}
Este relatório foi escrito no ambito da unidade curricular \ac{labi}, no qual tinhamos o objetivo de criar um sistema que permitiria criar músicas através da composição de vários pedaços de música(samples). Para o bom funcionar deste projeto deveriamos ter uma interface web, que iria ser desenvolvida em \ac{html}, \ac{css} e \ac{js}. 

De seguida também teriamos uma aplicação web feita em \textit{python}, onde teria de servir conteúdos estáticos(\ac{html}, \ac{css} e \ac{js}) e também métodos, métodos estes que têm de permitir o fluxo de informação entre os diversos componentes.

Como não podia faltar, temos de ter uma parte de persistência. Este componente é responsável pela armazenação de dados das respetivas musicas, bem como dados dos utilizadores. Para isto a base de dados utlizada foi \textit{SQLite3}

Para concluir, a parte do gerador de músicas. Esta era responsável por gerar as musicas que o utilizador criava.
\end{abstract}

\tableofcontents
% \listoftables     % descomentar se necessário
% \listoffigures    % descomentar se necessário


%%%%%%%%%%%%%%%%%%%%%%%%%%%%%%%
\clearpage
\pagenumbering{arabic}

%%%%%%%%%%%%%%%%%%%%%%%%%%%%%%%%
\chapter{Introdução}
\label{chap.introducao}

O objetivo deste relatório, é descrever e analisar os programas que foram desenvolvidos avaliando também a sua execução.

Conforme o programa, este documento encontra-se organizado por capítulos, sendo a introdução o primeiro. No \autoref{chap.implementacao} é descrito o funcionamento dos programas, sendo ainda separado em secções, consistindo na aplicação web, na base de dados e no site.

\chapter{Implementação}
\label{chap.implementacao}

\section{Interface web}
\label{sec.Interface web}


A interface web é constituída por vários campos, um para o nome da futura música("song name")e um para as batidas por minuto da música("BPM").
A interface também é constituida por uma tabela(4x12) onde se irá escolher a posição das samples. A primeira fila é para o som de um prato crash, a segunda é para o som de
um prato de choque(hit-hat), o terceiro para o som da tarola(snare) e por fim a ultima para o som de um bombo(kick). Por fim temos uma lista de efeitos que poderão ser escolhidos: delay, reverb, echo e tremolo.
\section{Aplicação web}
\label{sec.Aplicação web}
Consiste num javascript que: lê as entradas para o nome da música e para os bpm, utiliza um for para ler as posições de cada fila(samples), ou seja 4 for no total e 4 if's para
ler os efeitos selecionados. No fim converte tudo numa string com formato json para ser enviado para o cherrypy com a função .post.

\section{Base de Dados}
\label{sec.Base de Dados}

A base de dados, é responsável por guardar todas as informações relativas ás musicas, samples bem como os utilizadores e em que musica votaram. Assim a base de dados é responável por criar 3 tabelas, \textit{tabela1} (corresponde á tabela onde sao guardadas as informações dos samples), \textit{musicas} (guarda as informações das musicas), e a tabela \textit{users}.

No programa python relativo á base de dados, estão várias funções, funções estas que atualizam informaçoes quando um utilizador cria por exemplo uma musica, apaga uma musica e vota. Em cada função que criamos, temos de nos conectar á base de dados fazendo \textit{db = sql.connect('database.db')}. Depois disto utilizamos instruções do tipo \textit{insert, delete, select}  e \textit{update}. No fim destas instruções realizamos um \textit{db.commit()} para que a "transação" seja finalizada. Após todas estas operações finalizamos a ligação com a base de dados realizando um \textit{db.close()}

Para concluir, a base de dados é invocada pelo cherrypy cada vez que é precisa a sua utilização

\chapter{Conclusão}
\label{chap.Conclusão}
Não obtivemos um resultado final, ficou em falta o módulo do gerador de músicas e nao conseguimos interligar os diferentes componentes,apesar disso conseguimos por algumas coisas a funcionar(site,base dados,aplicação web).
\chapter{Contribuições dos autores}
\label{Contribuições dos autores}
Numa primeira fase tentamos abordar o projeto de uma maneira integral,mas com o passar do tempo,devido à dependencia de cada módulo, não obtivemos muitos progressos.Por isso decidimos dividir o trabalho em módulos independentes.
Neste projeto as contribuições dos elementos do grupo são apresentadas de seguida.

\begin{itemize}
\item António Domingues: Base de dados
\item Diogo Daniel: Interface web e aplicação web
\item João Mota: Gerador de músicas 
\item Pedro Alves: layout da pagina web(Home, listagem, about)
\end{itemize}

%%%%%%%%%%%%%%%%%%%%%%%%%%%%%%%%%
\chapter*{Acrónimos}
\begin{acronym}
\acro{ua}[UA]{Universidade de Aveiro}
\acro{miect}[MIECT]{Mestrado Integrado em Engenharia de Computadores e Telemática}
\acro{html}[HTML]{HyperText Markup Language}
\acro{css}[CSS]{Cascading Style Sheets}
\acro{js}[JS]{JavaScript}
\acro{labi}[LABI]{Laboratórios de Informática}
\acro{CSV}[CSV]{Comma Separated Values}
\acro{JSON}[JSON]{JavaScript Object Notation}
\end{acronym}


%%%%%%%%%%%%%%%%%%%%%%%%%%%%%%%%%
\printbibliography

\end{document}
